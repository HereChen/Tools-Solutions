\section{持续集成}\label{ux6301ux7eedux96c6ux6210}

\subsection{travis-ci}\label{travis-ci}

travis-ci 结合 github 使用, 每次提交自动执行编译或测试任务.

\begin{enumerate}
\def\labelenumi{\arabic{enumi}.}
\tightlist
\item
  \textless{}travis-ci.org\textgreater{} 用 github 账户授权登录
\item
  travis-ci 上添加需要持续集成的 github repo
\item
  github repo 中添加 \texttt{.travis.yml} 配置, 用以配置 travis
  所要执行的任务和环境
\item
  repo 更改完成, 提交, 可在 travis-ci 上实时查看任务执行过程
\end{enumerate}

备注:

\begin{itemize}
\tightlist
\item
  添加编译状态图标
\end{itemize}

\begin{verbatim}
[![Build Status](https://travis-ci.org/HereChen/Tools-Solutions.svg?branch=master)](https://travis-ci.org/HereChen/Tools-Solutions)
\end{verbatim}

\begin{itemize}
\tightlist
\item
  此库的 \texttt{.travis.yml} 样本
\end{itemize}

\begin{verbatim}
sudo: required
dist: trusty     # ubuntu 14.04

# install
before_install:
    - sudo apt-get update
    - sudo apt-get -y install texlive-full
    - sudo apt-get -y install pandoc

before_script:
    - chmod +x build.sh

script:
    - ./build.sh
\end{verbatim}

\section{Windows 下的 Linux
命令}\label{windows-ux4e0bux7684-linux-ux547dux4ee4}

\begin{longtable}[]{@{}cc@{}}
\toprule
\begin{minipage}[b]{0.09\columnwidth}\centering\strut
命令\strut
\end{minipage} & \begin{minipage}[b]{0.10\columnwidth}\centering\strut
资源\strut
\end{minipage}\tabularnewline
\midrule
\endhead
\begin{minipage}[t]{0.09\columnwidth}\centering\strut
curl\strut
\end{minipage} & \begin{minipage}[t]{0.10\columnwidth}\centering\strut
https://curl.haxx.se/dlwiz/?type=bin,
http://stackoverflow.com/questions/2710748/run-curl-commands-from-windows-console\strut
\end{minipage}\tabularnewline
\bottomrule
\end{longtable}

\section{工具安装}\label{ux5de5ux5177ux5b89ux88c5}

\subsection{Linux 下安装 nodejs
(x64)}\label{linux-ux4e0bux5b89ux88c5-nodejs-x64}

安装的方式包括: 从源码编译安装, 直接下载二进制, 依靠工具直接从库下载安装

\textbf{从源码安装}

\begin{verbatim}
curl -o node-v6.10.0.tar.gz https://nodejs.org/dist/v6.10.0/node-v6.10.0.tar.gz
tar -xvzf node-v6.10.0.tar.gz
cd node-v6.10.0
./configure
make
make install
\end{verbatim}

\textbf{CentOS 下载二进制安装}

\begin{verbatim}
# 64bit node install
curl -o node-v6.10.0-linux-x64.tar.xz https://nodejs.org/dist/v6.10.0/node-v6.10.0-linux-x64.tar.xz
mv /opt/
cd /opt/
tar -xvJf node-v6.10.0-linux-x64.tar.xz
mv node-v6.10.0-linux-x64 node

vi ~/.bash_profile
# 添加以下内容到 .bash_profile
# export NODE_HOME=/opt/node
# export PATH=$NODE_HOME/bin:$PATH
source ~/.bash_profile

node -v
\end{verbatim}

\textbf{Ubuntu 从库中下载安装}

\begin{verbatim}
curl -sL https://deb.nodesource.com/setup_6.x | sudo -E bash -
sudo apt-get install -y nodejs
\end{verbatim}

\subsection{git 安装}\label{git-ux5b89ux88c5}

\textbf{CentOS 从源码安装 git}

直接 \texttt{yum\ install\ git} 安装的版本比较老, 安装最新版可从源码安装

\begin{verbatim}
# 依赖工具安装
yum install curl-devel expat-devel gettext-devel openssl-devel zlib-devel
yum install gcc perl-ExtUtils-MakeMaker

# git 源码下载
# https://www.kernel.org/pub/software/scm/git/
curl -o git-2.9.3.tar.xz https://www.kernel.org/pub/software/scm/git/git-2.9.3.tar.xz
tar -xvJf git-2.9.3.tar.xz

# 编译安装 git
cd git-2.9.3
make prefix=/usr/local all
sudo make prefix=/usr/local install
\end{verbatim}

\begin{itemize}
\tightlist
\item
  \href{http://stackoverflow.com/questions/21820715/how-to-install-latest-version-of-git-on-centos-6-x-7-x}{How
  to install latest version of git on CentOS 6.x/7.x}
\item
  \href{https://git-scm.com/book/zh/v1/\%E8\%B5\%B7\%E6\%AD\%A5-\%E5\%AE\%89\%E8\%A3\%85-Git}{起步
  - 安装 Git}
\end{itemize}
